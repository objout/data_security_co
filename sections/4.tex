\documentclass[../main.tex]{subfiles}
\graphicspath{{\subfix{../figures/}}}

\begin{document}
\section{总结与发展趋势}
我们基于生成对抗网络(GAN)在联邦学习中设计了一种投毒攻击。
成功实施攻击的关键是在攻击者端部署一个 GAN 架构,
这个架构可以模仿其他参与者的训练数据集中的样本。
只要共享模型的准确性随着时间的推移得到提高,这个 GAN 的有效性就可以得到极大的保证,
这也是联邦学习的主要思想。此外,基于 GAN 的生成模型无法被检测到,
因为攻击者假装是联邦学习协议中的诚实参与者。
因此,这里提出的投毒攻击的有效性可以得到保证。

此外,现有的投毒攻击防御机制,如鲁棒损失和异常检测,
都不适用于联邦学习,因为它们都需要检测器访问参与者的训练数据和训练模型,
这与联邦学习的设计思想相矛盾。至于防御方面,
由于这里的投毒攻击依赖于 GAN 来模仿其他参与者的训练数据,
因此可以设计一种新的联邦学习框架,该框架隐藏全局模型的分类结果,
以防止攻击者使用 GAN 获取其他参与者的私有训练数据,
并最终防范内部参与者发起的投毒攻击。
%
\end{document}
